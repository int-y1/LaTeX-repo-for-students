\documentclass{article}
\usepackage{amsmath,amsfonts,amsthm}
\usepackage{enumerate}

\usepackage[english]{babel}

\newtheorem{lemma}{Lemma}

\usepackage[active, tightpage]{preview}
\setlength\PreviewBorder{1cm}

% weird hack from https://tex.stackexchange.com/a/98214
% this is needed to get the paragraph indents to appear
\usepackage{etoolbox}
\edef\keptparindent{\the\parindent}
\patchcmd{\preview}
  {\ignorespaces} %%% \preview ends with \ignorespaces
  {\parindent\keptparindent\ignorespaces}
  {}{}

\renewcommand{\thesection}{\arabic{section}.}
\renewcommand{\thesubsection}{\arabic{section}. (\alph{subsection})}
\renewcommand{\thesubsubsection}{\thesection (\alph{subsection})(\roman{subsubsection})}

\begin{document}

\begin{preview}
\section{Example}
\subsection{Example 1}
This template is useful if every single homework question requires you to upload a separate pdf file or image. You may even upload 20 separate files/images just to hand in your homework!

Bonus 1: When you're uploading a pdf file, does the server split the pdf file into individual pages? Can you find a way to generate an image for every page in a pdf?

Bonus 2: This template has very little whitespace. Less scrolling is needed!
\end{preview}

\begin{preview}
\subsection{Example 2}
\begin{lemma}
If $x < y$ then $-y < -x$.
\end{lemma}

\begin{proof}
Here is the proof:
\begin{align*}
x &< y \\
x+(-x-y) &< y+(-x-y) \\
-y &< -x
\end{align*}

It was trivial.
\end{proof}
\end{preview}

\begin{preview}
\section{Longer example}
Let $F(1,b)=1$ and $F(a,1)=1$. Also, for $a>1$ and $b>1$, let $F(a,b) = F(a-1,b) + F(a,b-1)$.

We show that for all $a \ge 1$ and $b \ge 1$, any set of $F(a,b)$ people would satisfy at least one of these conditions:

\begin{itemize}
\item There exists a crew of size $a$ that know each other
\item There exists a crew of size $b$ that don't know each other
\end{itemize}

The case where $a=1$ or $b=1$ is easy. The set is size $1$. Pick this person from the set, and make a crew.

Let's solve the case where $a>1$ and $b>1$. Assume that $F(a-1,b)$ and $F(a,b-1)$ are already proven.

Ask the first person. Put all of the known people into a set $S$, and all of the unknown people into set $T$. Note that $F(a,b) = 1 + |S| + |T|$ and $F(a,b) = F(a-1,b) + F(a,b-1)$. So we have $|S| + |T| = F(a-1,b) + F(a,b-1) - 1$. By the pigeonhole principle, we have $|S| \ge F(a-1,b)$ or $|T| \ge F(a,b-1)$. So we have two cases.

\begin{itemize}
\item Case 1: $|S| \ge F(a-1,b)$. We can use $F(a-1,b)$ to create two cases.
\begin{itemize}
\item Case 1a: We can find $b$ people in $S$ that don't know each other. In this case, we are already done.
\item Case 1b: There are $a-1$ people in $S$ that know each other. The first person knows everyone in $S$, so we can form a crew with the first person and the $a-1$ people in $S$. Then there is a crew of $a$ people that know each other, and we are done.
\end{itemize}

\item Case 2: $|T| \ge F(a,b-1)$. This case is similar to case 1.
\end{itemize}

The last step is to use induction to finish off the proof. This step is easy and is left to the reader as a short exercise.

Note: You may notice that $F(a,b) = \dbinom{a+b-2}{a-1}$.
\end{preview}

\begin{preview}
\section{Example of a long page}
\newcount\linenum
\linenum0
\loop\ifnum\linenum < 300
Long page.

\advance\linenum by 1
\repeat
In practice, homework solutions can fit on a smaller page.
\end{preview}

\end{document}
